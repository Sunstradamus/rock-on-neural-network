\section{Results and Conclusions}

Overall, we found that Lowe's SIFT algorithm has significant merit in object detection and classification as an existing, de facto approach and was well-suited to the problem when hands are involved due to their fairly unique and identifiable positions of curvature for use as keypoints. All of this led to an implementation that was relatively simple to get off the ground and have working. Additionally, after isolating for the hand, it did not require any additional training beyond a single-pass search of the input images and demanded very little computational power, allowing our approach to run, with backprojection, in realtime; however, due to the low overall correct classification rate of just over fifty percent, we ultimately preferred our neural network implementation. Our neural network, when trained for long enough, commanded as high as a ninety percent correct classification rate on our test data, and we know that this number can only be improved as more data and learning time could be provided. The primary factor acting against our neural network was the computational cost to compute the classification, preventing us from running it in realtime as we could with SIFT. Should we be able to improve the runtime, either with code improvements or additional hardware capabilities, the neural network approach would continue to become more attractive, despite its relatively higher cost of implementation.

\section{Possible Exploration and Expansions}

While the results of our expirements were pretty clear, with SIFT leading in performance and the neural network leading in accuracy, one of our primary avenues of future work would be to explore the data with more computational power available for our live demo, as this would almost certainly help to improve the runtime of the neural network, but we are unsure at this time by how much the improvement would be. Additionally, SIFT forced our hand into using backprojection to initially isolate for the hand within the capture, as background noise and mundane objects such as shirts had a significant detriment to its performance --due to hands having relatively few keypoints relative to their surroundings, despite their uniqueness. Due to this requirement, we have not explored using the neural network without backprojection, which would mean having the network learn the position of the hand in addition to just its classification. Conversely, we have also not explored using a more SIFT-focused dataset, wherein we would capture images of our hands in a purely isolated (e.g. black backgrounded) environment to remove the background noise when detecting keypoints in the hands, which would very likely improve SIFT's accuracy.